\section{学业相关}

\faq{有开学考试吗?}

有且仅有一门:本科生英语入学分级及免修资格考试。难度高于高考(以全国卷为标准),大概略难于四级。听力部分与四级一样在卷面上没有问题,许多同学可能不适应。

分级考试的前1000名可以在大一上学期(12月)即参加四级考试,其他同学至少在大一下才能考四级(以学校通知为准);前120名左右可以申请免修综合英语4,之后还要再通过免修考试才能免修(关于免修问题下文将详细解释)。按照有关规定,批准免修的人数不超过60人,19级实际成功30人,20级由于疫情原因没有这个考试,21级实际成功30人。

除此以外,本次考试将会作为英语能力分级的依据。不同等级将会在大一、大二学年体验到不同难度的英语课程。计算机学院(国家示范性软件学院)英语课程模式为“2+2+2+2”,即在大一上到大二下的四个学期内,每个学期均有2学分的英语课程(每周上2课时)。课程设置情况如下:

% 最后没有把垂直对齐改成居中对齐,是因为长表格可能有跨页
\begin{center}
    \begin{longtblr}[
        caption = 英语课程设置情况
    ]{
        colspec = cclcc,
        hlines,
        vlines,
    }
        层次 & 学期 & \SetCell{c} 课程名称 & 学分 & 周学时 \\
        \SetCell[r=8]{h} 基础 & \SetCell[r=4]{h} {第一学期 \\ 必修} & A 级:综合英语 4 & 2 & 2 \\
        & & B 级:综合英语 3 & 2 & 2 \\
        & & C 级:综合英语 2 & 2 & 2 \\
        & & D 级:综合英语 1 & 2 & 2 \\
        & \SetCell[r=4]{h} {第二学期 \\ 必修} & A 级:公众英语表达与沟通 & 2 & 2 \\
        & & B 级:综合英语 4 & 2 & 2 \\
        & & C 级:综合英语 3 & 2 & 2 \\
        & & D 级:综合英语 2 & 2 & 2 \\
        \SetCell[r=6]{h} {提高/发展 \\ 目标} & \SetCell[r=3]{h} {第三学期 \\ 必修} & A 级:学术英语入门 & 2 & 2 \\
        & & B/C 级:英语听说 2 & 2 & 2 \\
        & & D 级:综合英语 3 & 2 & 2 \\
        & \SetCell[r=3]{h} {第四学期 \\ 限定选修} & {A 级:下列课程八选一 \\(不含公众英语表达与沟通、\\学术英语入门)} & 2 & 2 \\
        % 下面这行多少沾点摆烂了,,实在不知道该怎么搞
        & & {B/C 级:下列课程十选一 \\ ★ABC级打通排课 \\ \quad ●专门用途英语类:\\ \quad ①科技英语阅读与翻译 \\ \quad ②商务英语与国际交流 \\ \quad ③学术英语入门 \\ \quad ④实用英汉翻译 \\ \quad ⑤思辨阅读与写作 \\ \quad ●跨文化交际类:\\ \quad ⑥跨文化交际英语 \\ \quad ⑦情景英语视听说 \\ \quad ⑧英美影视英语 \\ \quad ⑨英美文化概况 \\ \quad ⑩公众英语表达与沟通} & 2 & 2 \\
        & & D 级:综合英语 4 & 2 & 2 \\
    \end{longtblr}
\end{center}

\faq{我可以不购买教材吗?}

可以,教材购买是自愿的。你可以不买教材或者从学长学姐手中购买二手教材。学期初的统一征订将会享有一定的折扣(定价的88\%左右),可以按需订购。学期过程中很可能不能再从学校教材中心补买教材,不过可以选择网购。许多地方都可以找到电子版的教材,如果遇到不能到校上课需要远程教学的情况,老师一般也会提供电子教材。

\faq{我的成绩怎么计算?}

成绩主要分为两类,每学年评定的综合素质评价成绩(主要用于排名并发放每学年的奖学金)和在第6学期结束后评定的专业综合成绩(用于确定推荐免试攻读研究生(俗称保研)的名单)。

根据\href{http://my.bupt.edu.cn/content.jsp?urltype=news.NewsContentUrl&wbtreeid=1025&wbnewsid=95500}{北京邮电大学本科生综合素质评价办法(试行)(校发〔2021〕52号)}的相关规定,自2021级起:
\begin{center}
    \emph{学年综合素质评价成绩\\=基本素质成绩×10\%+专业素质成绩×70\%+发展素质成绩×20\%}
\end{center}

基本素质成绩=班级评议成绩-扣分项。班级评议成绩满分为100分,包括政治思想(25分)、学习态度(25分)、道德品质(20分)、法纪观念(15分)、健康生活(15分)。扣分项主要参考违规违纪、全校通报批评等。

专业素质成绩为课程学分加权成绩,包括必修课(含体育课)和专业选修课程(按首次成绩计算),不含全校任选课、辅修课程。所谓加权成绩即所有涵盖的科目成绩按照学分权重计算平均分,免修英语不计入内。简单来说,学分越高的科目,权重越高。

发展素质成绩(俗称德育分)每学年通过年级制定的统一标准进行评定。无不良记录、参加志愿活动、实践活动和在学生组织担任职位有助于提高德育成绩。具体需依照当年的德育分评定细则,但大致组成框架如下:

\begin{center}
    \begin{longtblr}[
        caption = 德育分组成参考
    ]{
        colspec = {X[2, c] X[3, l] X[1, c] X[6, l]},
        hlines,
        vlines,
    }
        类别 & \SetCell{c} 项目 & 占比 & \SetCell{c} 主要内容 \\
        \SetCell[r=3]{h} 身心健康成绩 & 体能测试成绩 & 20\% & 体质健康测试 \\
        & 体育锻炼成绩(健步跑) & 10\% & 在运动世界校园中每学期的跑步公里数 \\
        & 身心健康活动评价成绩 & 10\% & 参与身心健康活动等 \\
        \SetCell[r=5]{h} 其他发展素质 & 思想成长 & 14\% & 参与主题教育活动,集体荣誉等 \\
        & 学术创新 & 14\% & 参与学术类活动及相关荣誉 \\
        & 美育素养 & 12\% & 参与美育类活动及相关荣誉 \\
        & 劳动素养 & 12\% & 社会实践,社团公益等劳动活动 \\
        & 组织协调能力 & 8\% & 党团班学(含学生社团)组织工作 \\
    \end{longtblr}
\end{center}

根据\href{http://my.bupt.edu.cn/content.jsp?urltype=news.NewsContentUrl&wbtreeid=1036&wbnewsid=95475}{北京邮电大学推荐优秀应届本科毕业生免试攻读研究生管理规定(校发〔2021〕50号)}的相关规定,自2020级起:
\begin{center}
    \emph{专业综合成绩=专业成绩(100分)+实践活动加分(上限4分)}
\end{center}

专业成绩的计算方式类似于奖学金中的“专业素质成绩”,即必修课、专业选修课的课程加权平均分,俗称智育成绩。

实践活动包括学术类竞赛、科研实践活动、体育类竞赛、文艺类实践活动以及退伍复学五个方面,成果及其对应加分,由学院根据北京邮电大学关于本科学生参加各类实践活动认定的实施细则(校发〔2021〕51号)的相关规定进行细化。

推荐免试攻读研究生(保研)除按照专业综合成绩排序外,还有一些基本条件:
\begin{itemize}
    \kaishu
    \item 前三年综合素质评价平均成绩须在其专业60\%(含)以内
    \item 前三年体质测试平均成绩须达到60分及以上
    \item 前三年完成劳动教育学时不低于24学时(该学时计算截止时间为每年8月31日)
\end{itemize}

\faq{什么是培养方案?我怎么看培养方案?}

培养方案是学校为每个专业制定的培养目标(最低毕业标准)。里面包括了你本科期间需要修读的课程,参与的实践活动和其他详细事项。培养方案会在开学后由辅导员发给大家,厚厚一本,也可以找辅导员要电子版。拿到培养方案后一定要仔细研读!

虽然很多地方都和你没太大关系,但也有很多需要注意的地方:首先你要对你每个学期所需要的学习的课程有一点了解,培养方案中有一张流程图,几乎是非常详尽的画出了每个学期的理论课和实习课。如果不出意外情况,课表就会根据培养方案来安排。重点关注一下学分比较高的课程,这些都是不能挂的重要科目。然后你可以找到任选课表和专业选修课表,提前规划一下你的选修课。此外还可以找到每个院的辅修安排,如果你大二以后想辅修一个专业,可以先看一看。(然而不建议辅修,建议专精一项)

\faq{什么是德育分?德育分怎么拿?}

德育分,或者说德育成绩,主要是指代学生除智育、体育素质外的其他综合素质。在2021级起的综合素质评价里主要表现为:基本素质成绩、身心健康活动、其他发展素质(这些名词的含义可见前文)。

基本素质评价几乎相持由平时的各项评价指标决定,包括青年大学习的完成率、打卡缺勤的次数、宿舍查寝的分数等,还有一部分是由班级同学匿名打分得来。(班级评议成绩=学生自评成绩×10\%+班级同学互评成绩×70\%+辅导员、班主任评议成绩×20\%)

而发展评价则“各凭本事”(论参加活动体现在何处)。因此要想德育分得分高,需要积极主动地参与各类活动(尤其是院内)(虽然不鼓励过于功利地参加活动,但活动之间确实存在有“认定”与“不认定”的情况,只能说本学院主办的活动具有天然优势)

德育分会在每学年秋季学期开始时评定(即大二的秋季学期评价大一学年的德育分,以此类推),奖学金的评定也是如此。(在口口相传过程中,经常会出现“德育分”和“综合素质评价成绩”不分的情况,比如上文)

\faq{奖学金怎么评?其他评优评奖呢?}

在秋季学期,综合素质评价成绩认定后,会开展一次本科生评优表彰工作。本科生评优表彰工作的基本条件是:无不及格课程成绩(必修课、专业选修课、体测)。

奖学金评定一般依照当年综合素质评价成绩依次选取(不重复),实际操作过程中一般会公示每个人获得什么奖学金,然后在学工系统中申请相关奖学金、荣誉称号。

\begin{center}
    \begin{longtblr}[
        caption = 奖学金与荣誉称号情况
    ]{
        colspec = {X[2, c] X[3, l] X[2, c] X[2, c]},
        hlines,
        vlines,
    }
        类别 & \SetCell{c} 奖项名称 & 名额 & 金额 \\
        \SetCell[r=6]{h} 奖学金 & 国家奖学金 & 1\% 左右 & 8000 \\
        & 学校一等奖学金 & 3\% & 5000 \\
        & 学校二等奖学金 & 10\% & 3000 \\
        & 学校三等奖学金 & 25\% & 1000 \\
        & 各类企业奖学金 & 以通知为准 & {8000/5000 \\ /3000/1000} \\
        & 国防奖学金 & 符合评选条件 & 按规定执行 \\
        \SetCell[r=2]{h} 奖学金(注册困难生专项) & 国家励志奖学金 & 依教育部下发名额而定 & 5000 \\
        & 企业奖学金 & 以通知为准 & 3000/1000 \\
        \SetCell[r=6]{h} 荣誉称号 & 三好学生 & 10\% & 500 \\
        & 优秀学生干部 & 5\% & 500 \\
        & 文体积极分子 & 5\% & 300 \\
        & 学习进步奖 & 3\% & 300 \\
        & 本科生先进班集体 & 20\% & 奖状一张 \\
        & 本科生优秀学生宿舍 & 10\% & 奖状一张 + 小奖品(生活用品等)
    \end{longtblr}
\end{center}

除了个人荣誉之外也有一些集体荣誉,如本科生优秀班集体和本科生优秀学生宿舍等。以上各类荣誉的评选方式届时会具体通知。

\faq{选修课怎么选?我该选什么课?}

每学期都会有任选课的机会,选修会在学期开始前在教务系统上\sout{抢}选课,选修课程表会提前放出,任选课主要可以分为:
\begin{itemize}
    \kaishu
    \item 理科类:数学物理有关的课程
    \item 工科类:各个院开设的专业相关选修课
    \item 艺术类:音乐美术电影相关的课程
    \item 人文社科类:政治、经济、历史、语言等相关课程
    \item 体育类:任选的体育课
\end{itemize}

除了体育类以外,本科期间剩下四类中与你所学专业无关的类别每个人都必须至少修读一门。以软件工程专业为例,则艺术类与人文社科类各需要至少修读一门。具体要求请参考你的培养方案。这个政策每年也可能会改变。如果你已经修满了所需学分,就可以不再选修任选课。

以上课程中部分是在智慧树等网课平台上课,这类网课名额多,没有抢课压力,不用按时上课,刷视频线上答题就能过,但内容比较水,是刷学分利器。由于疫情,大部分其他课程也都改成了线上授课,一般都是在腾讯会议等平台直播授课。这些(本应线下授课,但由于疫情改成线上授课的)线下课有的要期末考试,有的期末只要完成相应考核任务(写小论文之类)就可以给分。

选择任选课先以修满学分为目标,每个类别里面选一些你有兴趣的,然后可以咨询一下学长学姐课程的质量如何,考试难易程度如何,还要看看运气能不能抢到。你可以在一些地方看到总结好的公选课避雷指南,抄作业即可。选修课的具体开课列表常常变动,有的课每年作业量及给分也会有不小的变化,此部分经验进攻参考。抢不到课或者上课后发现不喜欢,在开任选课两周后有一次退补选机会。这里说几个热门的课程:

高等数学解题方法:仅大一开课,枯燥但最有用,每学期开两个班,基本爆满。考试前没有选的也去蹭课,但要考试。建议不选然后去蹭课。

公共日语、公共法语:同样爆满\sout{,但必须大二以后才能选},课时是其他选修课的两倍(也就是一周上两次),但学分也是两倍。

诺贝尔物理学奖史话:首次使用全息投影双校区同时授课的课程。

MATLAB应用:秒空课程之一,比其他几个语言啥的都快。

乐理基础:避坑课程,如果没有基础考试极易挂科(本条来自学长学姐告诫)

需要注意的是,任选课的成绩是不计入综合排名和保研排名的,但会出现在申请国外高校时需要使用的成绩单上,影响你的GPA。如果成绩实在不理想可以干脆挂科,只要后续不再选修这一门任选课,挂科记录就不会出现在成绩单上。

除了任选课以外还有专业选修课和英语、体育选修课(大二开始才有),具体参考你的培养方案。这些选修课都必须修满对应学分,基本上是二选一三选一系列,记得千万不要忘了选。

\faq{课表的时间安排是怎么样的?}

上午最早课程是8:00,最后一节12:15下课(1到5节)

下午最早课程13:00,最后一节18:10下课 (6到11节)

晚上课程最早18:30开始,最晚20:55下课(晚课都是任选课,且实际上一般从19:20开始上课,不用担心没时间吃晚饭)

中午休息时间较短,如果碰到五六节联排会很难受。但这样为大家争取到了更长的假期……(目前看来2021级软件工程只有19-20班的星期四有这种困扰)

周三下午全校无课,时间可以用来进行各种实践活动(开各种会)

原则上周末不上课。

\faq{上课的安排是怎样的?}

大班课:如数学课、物理课等公共课,一般以大班为单位,一百多人在一个大教室里上课。

小班课:一般是专业课,一两个班级在一个小教室里上课。2020级是16-18班一起,19-20班一起。也有部分课程16-17班一起上课,18-19班一起上课,20班单独一个老师。对于21级而言,由于双培生单独成班,所以一般是16-18班一起上课,19-20班以及双培班一起上课。

特殊的课程:英语课分级以后单独编班,一个班30人一起上课,听说课则是一个班在听力机房上课;各种上机实验课也都在机房上课(但一般你还是带自己的笔记本,最好不要指望用学校的电脑),还有一些课程在特殊的实验室上课。

\faq{作业写在哪里?怎么交?}

一般来说,作业有这么几种:

课堂派等线上平台布置的小作业:根据老师要求可以写在纸上拍照上传,有时也可以提交电子文档,ddl(截止日期)由老师设定,超时就算作缺作业(不过老师们一般都比较仁慈,补交也不是不行)

课前手动收的作业:一般写在纸上(也可以写电子版,具体看老师是否有明确要求),和你在初中高中写的作业差不多,由学委或老师指定的课代表统一收取。

上机作业:又称OJ作业,即线上编程测试,只有部分专业课有,这个到时你们就知道了。

大作业:一部分的课程会布置占成绩比重较高的大作业,有的个人完成,也有的以小组完成。这类作业会有详细的完成要求及提交说明,一般成品通过邮件提交。有的课程期中或期末并不考试,也以大作业形式完成考核。

哦对了,英语听说课则需要在专门的新理念平台上做线上答题作业,由于这个平台是在太拉胯了,你的电脑可能不会兼容,这时建议每两周抽一次没课的上午去上课的机房完成(当然有很多办法可以在寝室完成(甚至让脚本自动完成)该作业,具体的到时请咨询你身边的大佬)


\faq{每门课的最终成绩怎么算?翘课对成绩影响大吗?}

因课而异,一般是平时成绩+期中+期末,期末要占到40\%以上。

每个老师对平时成绩的定义不一,有的要求考勤+作业完成,有的只要求作业,有的甚至没有要求。但是尽量不要翘课,平时成绩好一点有助于你期末考试失利的时候老师拉你一把。

\faq{如何免修课程?免修的课程如何计算成绩?}

北邮现在不允许免修除了大一上学期的综合英语4以外的其他科目。要免修综合英语4,首先需要在入学时的分级考试中取得一个相当靠前的成绩,然后报名参加英语免修考试,经考核合格后即可获准免修此课程。免修的课程不计入最后计算的成绩,在成绩单上显示为“免修”。

以前也曾允许过免修其他课程,目前暂时没有恢复相关制度的消息。

\faq{从哪里可以获得学习资料以及历年的试卷?}

学习资料:获得途径有专业群、班级群、各个科目老师建立的课程群等等。你也可以从好朋友或者学长学姐那里取得一些秘藏资源。(在BYRBT上有很多其他学校的好资源,关于BT站的介绍见“内网资源”一节)

历年试卷:由于\sout{懒得出题}众所周知的原因,往年的试卷通常是不会外传的(除了高等数学、线性代数等科目)。不过你可以询问你的学长学姐,他们有时会保留往年的试卷,也可以去打印店碰碰运气(一般打印店老板会保留每年同学们经常打印的资料)。或许还有别的一些奇怪的渠道能获得。

本群的群文件中有着非常丰富的学习资料,基本能涵盖你想要的所有,需要请自取。

\emph{需要注意的是北邮没有任何学生组织、大创项目、微信小组等“提供”或售卖高数、线代等热门课程的“内部”或“特供”资料,在QQ群和微信群等社交网络上宣传资料的陌生人均为社会组织或诈骗团伙,一定不要轻易上当!}

\faq{北邮的假期多吗?}

北邮江湖人称北京不调休大学,很多时候节假日都会连放很多天,即别人放假的时候我们放假,别人调休的时候我们还是放假。例如2022年的国庆,如果没有意外情况,大家可以享受到⑨天的超长假期。

\faq{我想转专业,应该怎么做?}

根据\href{http://my.bupt.edu.cn/content.jsp?urltype=news.NewsContentUrl&wbtreeid=1036&wbnewsid=25646}{北京邮电大学本科生校内转专业办法(校发〔2020〕14号)}的相关规定,符合条件的本科生可以在大一下学期初和大二上学期初提出转专业申请。转专业考核仅由接受转入学生的学院进行,各学院要求不同。一般由已取得的课程成绩和可能的笔试、机试成绩按比例计算出综合成绩,排名靠前的经面试后就可以成功转专业了。对于软件工程专业来说,想要转出的同学会比较少,更详细的信息可以咨询群里的学长学姐。
